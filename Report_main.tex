
\documentclass[11pt]{article}

\usepackage{graphicx}       % to include graphics
\usepackage{hyperref}       % to simplify the use of \href
\usepackage{amsmath}
\newtheorem{lemma}[theorem]{Lemma}

\title{Report on knotted protein stretching}
\author{Pawel Dabrowski-Tumanski}
\date{\today}

\begin{document}
\maketitle

%%%%%%%% Introduction %%%%%%%%
\section{Introduction}
\label{sec:intro}
In this report, the stretching of three proteins is analyzed -- TrmD, Tm1510 and their fusion.
TrmD is the $+3_1$-knotted tRNA (guanine-N(1)-)-methyltransferase, from \textit{Haemophilus influenzae} bacteria, classified as a member of SPOUT Methyltransferase superfamilly.
Tm1510 is the protein of unknown function from \textit{Thermotoga maritima} bacteria.
Due to its sequential similarity (?) it is also putatively $+3_1$-knotted protein.

The stretching data analyzed in this report is of three sources -- the coarse-grained simulations, the all-atom simulations, and the optical tweezers experiment.
In all the cases, the proteins are stretched in a constant-speed fashion.
The aim of this report is first to determine the topology of stretched proteins.
Second, to understand and describe the whole stretching process done \textit{in silico}.
This leads to the third goal which is describe and understand the origin of the stretching curve obtained experimentally, with the identification of the intermediate products, if possible.
The ultimate goal is to select the theoretical models which describe the stretching most consistently with the experiment.
The analysis includes the lengths of stretched proteins, as well as the thermodynamics of the process.

The report is constructed as follows.
First, the theoretical description of observed phenomena is described (sec.\ref{sec:theory}).
Next, the methods section (\ref{sec:methods}) include the details of proteins, their sequences, expression, the experiment and simulation used.
The result section (\ref{sec:results}) include all the information obtained from the analysis, subsequently discussed in the last (sec.\ref{sec:discussion}) section.

%%%%%%%% Theoretical description %%%%%%%%
\section{Theoretical description}
\label{sec:theory}
This section contains the information about the theoretical backgrounds of the analysis performed.
In particular, it describes the force response of the protein chain upon stretching (the WLC model -- sec.\ref{subsec:wlc}) with its transformations (sec.\ref{subsec:inverting}), and the analysis of the thermodynamics of the process (sec.\ref{subsec:thermodynamics}), including the Dudko-Hummer-Szabo equation (sec.\ref{subsec:dudko}).
Finally, the information about the knot topology on the stretching is described (sec.\ref{subsec:knotted}).

\subsubsection{Force response - the WLC model}
\label{subsec:wlc}
The stretching of the protein chain effects in a force response described usually by the Worm-Like Chain model (WLC model).
The force $F(d)$ as a result of protein extension $d$ is dependent on the temperature $T$, protein persistence length $p$, and the contour length $L$ of the chain, which can be stretched as in eq.\ref{eq:wld}:

\begin{equation}
    \label{eq:wlc}
    F(d) = \frac{k_{B}T}{p}\left(\frac{1}{4(1-\frac{d}{L})^2}-\frac{1}{4}+\frac{d}{L}\right)
\end{equation}

where $k_B$ is Boltzman constant.
The meaning of the contour length $L$ is presented in Fig.\ref{fig:theory_contour_length}.
For sufficiently high force, some bonds may be thorn apart, changing the contour length upon rupture (Fig.\ref{fig:theory_contour_length}).
As a result, a sudden jump in protein force response is obtained.

\begin{figure}
    \includegraphics{theory_contour_length.png}
    \caption{}
    \label{fig:theory_contour_length}
\end{figure}

\subsection{Inverting the WLC model}
\label{subsec:inverting}
As the temperature and the persistence length are constant during stretching, and extension $d$ is necessarily smaller than contour length $L$, the WLC model of eq.\ref{eq:wlc} may be simplified to:
\begin{equation}
    \label{eq:wlc_simple}
    F(x) = p_T\left(\frac{1}{4(1-x)^2}-\frac{1}{4}+x\right)
\end{equation}

with $0\lt x \lt 1$ and $p_T=\frac{k_{B}T}{p}$.
Such equation can be used to invert this relation.
In particular, one can rearrange the eq.\ref{eq:wlc_simple} into:
\begin{equation}
    \label{eq:wlc_polynomial}
    x^3 - (\frac{9}{4}+\frac{F}{p_T})x^2 + (\frac{3}{2}+\frac{2F}{p_T}) - \frac{F}{p_T} = 0
\end{equation}

\begin{lemma}
    The function $y(x) = x^3 - (\frac{9}{4}+c) x^2 + (\frac{3}{2}+2c) x - c$, has exactly one root in the interval $(0,1)$ for any $c \gt 0$.
    \\\textbf{Proof}\\
    Observe, that
    \begin{multiline}
        y(0) = -c < 0 for c > 0\\
        y(1) = \frac{9}{4}
    \end{multiline}
    This shows, that there is at least one root $x_0$ with $0 \lt x_0 \lt 1$.
    If there are three distinct roots in this interval, the function $y$ has also to extrema -- local minimum $x_{MIN}$ and local maximum $x_{MAX}$, fulfilling $0 \lt x_{MAX} \lt x_{MIN} \lt 1$.
    Both extrema are the roots of the derivative:
    \begin{equation}
        y'(x) = 3x^2 - (\frac{9}{2}+2c)x + (\frac{3}{2}+2c)
    \end{equation}
    Observe, however, that $y'(1)=0$, therefore, at least one extremum is is not located inside the interval $(0,1)$.
    This shows, that the assumption of the existence of three roots in the interval $(0,1)$ is wrong, and therefore, there is exactly one root in this interval.
\end{lemma}

By the lemma, there is only one $x(F,p)$ being the physical inverse of the eq.\ref{eq:wlc_simple}.
Let us denote such solution as $x_F$.
This allows, for known $d,F(d),p$ calculate $L(d,F(d),p)=d/x_F$

\subsection{Thermodynamics of transitions}
\label{subsec:thermodynamics}

\subsection{Dudko equation}
\label{subsec:dudko}

\subsection{Stretching of knotted polymer}
\label{subsec:knotted}

%%%%%%%% Methods %%%%%%%%
\section{Methods}
\label{sec:methods}
\subsection{Protein sequences and structures}
\label{subsec:sequences}
TrmD - PFAM PF01746, UNIPROT P43912
Tm1510 - PFAM PF06245, UNIPROT Q9X1J8
\subsection{Protein expression}
\label{subsec:expression}

\subsection{The theoretical models}
\label{subsec:models}

\subsection{The experimental setup}
\label{subsec:experiment}

\subsection{The code for the analysis}
\label{subsec:code}

%%%%%%%% Results %%%%%%%%
\section{Results}
\label{sec:results}

%%%% CG %%%%
\subsection{Analysis of coarse-grained simulations}
\label{subsec:cg}
%%%% All atom %%%%
\subsection{Analysis of all-atom simulations}
\label{subsec:aa}
%%%% Experiment %%%%
\subsection{Analysis of experimental data}
\label{subsec:exp}

%%%%%%%% Discussion %%%%%%%%
\section{Discussion}
\label{sec:discussion}

%%%%%%%% References %%%%%%%%
\bibliography{biblio} 
\bibliographystyle{ieeetr}
\end{document}  
