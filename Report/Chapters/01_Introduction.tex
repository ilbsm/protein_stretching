% Chapter 01 - Introduction
\chapter{Introduction}
\label{ch:intro}

\dropcap{T}he aim of this project was to analyze the experimental and theoretical stretching curves of three proteins - TrmD, Tm1570, and their fusion.
As the proteins analyzed were supposed to be knotted, the first important task of the analysis was to establish the actual topology of the proteins.
On the other hand, the simulated stretching was performed in various models.
Therefore, the second important task of the analysis was to select the model, which matches the experiment best.
Finally, the third task was to explain the unfolding pathway observed in the experiment, based on the results of the simulated stretching.

Usually, analysis of the protein stretching is done by fitting the stretching curves and comparing the parameters obtained.
However, it quickly turned out, that the standard methods cannot be applied in this case.
In particular, usually it is assumed, that the protein is much more rigid than the DNA linker, and therefore in the first stage of stretching only the DNA gets extended.
This is, however, not the case of TrmD, which is composed of two subdomains connected with the flexible linker.
Alternatively, one knows the initial distance between the termini of protein from crystal structure.
This time, however, there in case of TrmD and Tm1570 there are only the crystal structures of homologs, not the structures measured.
As the initial topology is unknown, the end-to-end distance taken from this crystal is not a reliable value.

\graffito{Contour length is the largest part of the chain, which can currently be stretched.}

On the other hand, until now, no thorough \textit{in silico} analysis of protein stretching was done so far.
Usually, the general scheme of the unfolding is analyzed.
However, to select the best model, the coarse pattern of the stretching curve, emerging from simulations is not informative enough.
This creates need of in-detail analysis of the theoretical stretching, with calculation of the contour lengths.
The contour lengths are then the numbers, which can be compared with experiment and the model, which recreates these numbers best can be considered as the best model.
However, such meticulous analysis of simulated stretching was not performed yet.

Therefore, apart from the raw results of the analysis of stretching, in this work many new techniques were introduced to solve previously unknown problems.
In particular, the report:
\begin{enumerate}
    \item introduces a new model of of protein stretching (stretch-adjusted WLC - Sec.\ref{subsec:results-approach-elasticity}), which is more reliable than usually used eWLC;
    \item introduces the method of inverting the force-distance dependence (Sec.\ref{subsec:results-approach-inverting});
    \item establishes the protocol of the analysis of the simulated stretching of proteins (Sec.\ref{subsec:results-approach-theory});
    \item introduces a way to deal with stretching of proteins with flexible linker (Sec.\ref{subsec:results-approach-linker});
    \item introduces a new method of extracting the proteins persistence length, which is then used to establish the topology of the models.
\end{enumerate}

The analysis

In particular, the ability to estimate the total protein length allowed to determine the topology of the protein.
Surprisingly, not all the proteins were shown to be knotted.
This phenomenon is treated in greater detail in further parts of the report.

\graffito{PyPi is the standard source of Python package.}
During the analysis, many new techniques (described in Chap.\ref{ch:methods}) were introduced.
These techniques were not implemented anywhere else so far.
Therefore, to facilitate the work, they were written from scratch.
As a result, a byproduct - the Python3 package called StretchMe was created.
The fully functional package can be download and installed from PyPi server.

The report is organized as follows:
\graffito{WLC is the Worm-Like Chain model, used commonly in the description of protein stretching.}
Chapter\ref{ch:theory} contains the survey of the most important information about mechanical stretching of proteins.
It includes the short review of the literature concerning stretching of proteins with complex topology, description of the WLC model, the effect of the knot on the stretching curves, and more advanced ways of the analysis, such as Dudko-Hummer-Szabo analysis of states lifetime.

Chapter\ref{ch:methods} contains information about the methods used.
In particular, within the chapter, the proteins investigated are described along with the experimental setup.
It also contains th the information on different maps used in \textit{in silico} analysis and new techniques introduced in this work.
Finally, it contains the short description of the StretchMe package.

Chapter\ref{ch:results} contains the results of the project.
In particular, the topological analysis of the proteins is described.
It also contains the description of the theoretical results, along with the comparison with the experiment.
As a result, the most promising models are indicated for further analysis.
Finally, based on the models chosen, the possible experimental unfolding pathway of TrmD is described.

Chapter\ref{ch:discussion} contains the discussion of the results, their possible causes and the proposition of further experiments which can be done to answer the new questions.

The report ends with a set of appendices, which include the exact fitted parameters, derivation of most important formulas and the exact form of functions used in StretchMe package.