% Chapter 01 - Introduction
\chapter{Introduction}
\label{chap:intro}

The aim of this project was to analyze the experimental and theoretical stretching curves of some knotted proteins.
As a result of the analysis, the model best matching the experimental data had to be selected.
Based on the result of this model, the states visible in the experimental data were to be described.

Quickly it turned out, that the analysis of the data is much more complex than in the standard case.
In particular, in case of experiment, the protein gets stretched simultaneously with DNA linker, which is assumed not to happen in standard analysis.
Moreover, it turned out, that the protein may not feature the knot in every case.

In case of the analysis of the simulated stretching, the meticulous analysis of the curves revealed the need to include the elasticity of the bonds between the residues, which was not applied previously.
The more detailed analysis showed, that the commonly used method of including the elasticity (eWLC method) gives some mismatch with the data observed.
To fix this, alternative method, called stretch-adjusted WLC was introduced.

As a byproduct, the analysis resulted in a software called StretchMe Python 3 package, which is capable of analysing such cases.

This report describes the theory of protein stretching, the obtained results and possible ways of improving further results in the discussion.
It also contains all the plots obtained, derivation of some equations, parameters used, and exemplary code needed to recreate the plots in the Appendix section.