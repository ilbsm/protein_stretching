% Chapter 03 - Methods
\chapter{Methods}
\label{ch:methods}

%%%% proteins %%%%
\section{The proteins analyzed}
\label{sec:methods-proteins}
In the experiment, three proteins were investigated:
\begin{enumerate}
    \item TrmD, $+3_1$-knotted tRNA (guanine-N(1)-)-methyltransferase, from \textit{Haemophilus influenzae} bacteria, classified as a member of SPOUT Methyltransferase superfamilly;
    \item Tm1570, the protein of unknown function from \textit{Thermotoga maritima} bacteria, putatively $3_1$ knotted protein;
    \item Their fusion TrmD-Tm1510, which is expected to have a composite knot $3_1 \# 3_1$.
\end{enumerate}
The sequences of TrmD and Tm1570 are presented in Tab.\ref{tab:methods_sequences}:
\begin{table}
\begin{tabular}{ll}
TrmD & 240 residues\\
1   & MKRYNVITIFPEMINEIFKYGVLSKGIDIGLFRVNPINLRDYTEDKHKTVDDYQYGGGHGLVMKPEPIYKAIADLKSKKD\\
81  & THVVFLDPRGEQFTQKTAERLYNYDDITFVCGRYEGIDDRVRELMADEMISIGDFVITGGELAAVTIIDAVARLIPGVLG\\
161 & DENSPNEESFTTGLLEYPHFTRPAEFMGKKVPEVLISGNHEEIRRWRLTESIKTTLQNRPDMILRKSLSREEEQILWSLT\\\hline
Tm1570 & 193 residues\\
1   & RGVQRKYNIYVALMHYPMRDKEGKVVTTSITNMDLHDISRSCRTFGVKNYFVVNPMPAQREIASRVVRHWIKGYGATYNE\\
81  & NRKEAFEYTIITDSLASVIKSIEEKESGSPIIIATTARYQQKAISIEKLKEIADRPILLLFGTGWGFVDDILEFADYVLK\\
161 & PIHGVGDFNHLSVRSAVAIYLDRINRSFQEDIL\\\hline
\end{tabular}
\caption{The sequences of analyzed TrmD and Tm1570.}
\label{tab:methods_sequences}
\end{table}

No crystal structure for TrmD and Tm1570 exists, therefore, for the theoretical analysis they were modelled using I-Tasser server, based on homologous structures with PDB codes 5wyr (for TrmD) and 3dcm (for Tm1570).
Similarly, the fusion TrmD-Tm1570 does not possess crystal structure.
It was modelled by connecting the basic structures manually.
The modeled structures overlaid with their templates are shown in Fig.\ref{fig:methods-overlaid}.

%\begin{figure}
%    \includegraphics[width=\linewidth]{methods_overlaid.png}
%    \caption{The overlaid modeled proteins with their templates. Left TrmD and structure with PDB code 5wyr, right Tm1570 and structure with PDB code 3dcm.}
%    \label{fig:methods_overlaid}
%\end{figure}

The exact fusion of two proteins has 433 residues (240 + 193).
However, in the modelled structure, the chain has 432 residues, with the last C-terminal leucine missing.
Probably that is unintended omission, therefore in the interpretation of experimental data, the protein was still considered as 433-residue long.
On the other hand, due to the experimental setup (Sec.\ref{sec:methods-setup}), cysteine residues were added to both ends of the chain.
Therefore, the total length of the residues in the experiment was 242 (for TrmD), 195 (for Tm1570) and 435 (for TrmD-Tm1570).

The lack of the crystal structure prevents calculating only the differences in the contour length during protein stretching, as there is no reference point (initial end-to-end distance).
Using the distance from the homologous structures is not reliable, as the topology of target is unknown.
For example, if the target is slipknotted, its end-to-end distance may be significantly different than the in the knotted template (Fig.\ref{fig:method-slipknot}).
Moreover, in case of the fusion, there is no template the take the end-to-end distance from.

%\begin{figure}
%    \includegraphics[width=\linewidth]{methods_slipknot.png}
%    \caption{}
%    \label{fig:methods-slipknot}
%\end{figure}


%%%% experimental setup %%%%
\section{The experimental setup}
\label{sec:methods-setup}
The experiment was performed as in\cite{jahn2014charged}.
In particular, the terminal protein cysteines were connected with 185nm DNA strands by utilizing the ``click'' chemistry approach of thiol-maleimide reaction (Fig.\ref{fig:methods-click}).

%\begin{figure}
%    \includegraphics[width=\linewidth]{methods_click.png}
%    \caption{}
%    \label{fig:methods-click}
%\end{figure}

The DNA strands were in turn fictionalized to connect to $1\mu m$ beads, trapped in optical tweezers.
The experiment was performed by moving the beads apart with constant velocity, until the force reached about 40pN.
Next the protein was relaxed (no force applied) for a fixed amount of time.
The protein stretching speed: $v=500nm/s$.
The temperature of measurement: $t=28^{o}C$.
The spring constant of the optical tweezers: $0.3 pN/nm$.

%%%% maps %%%%
\section{The theoretical setup}
\label{sec:methods_maps}
In the simulations, different structures of homological proteins were used:
\begin{enumerate}
    \item the best model of TrmD from I-Tasser;
    \item knot-less structure with the sequence of TrmD;
    \item slipknot structure with the sequence of TrmD;
    \item TrmD homologue with PDB code 5wyr.
\end{enumerate}

The slipknot structure was not analyzed in this report, as the analysis was left for Aleksandra Gajkowska.
The stretching was performed within various conditions, differing in the contact map used, the temperature, and the spring constant.
The contact map used in simulations:
\begin{enumerate}
    \item $C\alpha$ map, where the amino acids are in contact if their $C\alpha$ atoms are in vicinity of $6AA$ cutoff (denoted by capital letter `C' in further analysis);
    \item Tsai map, where the contact between residues is specified based on the overlapping of the van der Walls spheres of heavy atoms (denoted by `P');
    \item Shadow map, where the contact is specified by the cutoff, subjected to the condition, that one residue cannot shadow the other contact (denoted by `S').
\end{enumerate}

Depending on the temperature and the elasticity of the spring, the models used were named with two-letter code, described in Tab.\ref{tab:methods-maps}.

\begin{table}
    \begin{tabular}{c|c|c|c||c|c}
        Map & $T=0.5$ & $T=0.6$ & $T=0.4$ & $T=0.5$, $\kappa_S=0$ & $T=0.3$ \\\hline
        $C\alpha$ & CA & CB & CC & CD & CE\\\hline
        Tsai & PA & PB & PC & PD & ---\\\hline
        Smog & SA & SB & SC & SD & ---\\\hline
    \end{tabular}
    \caption{The names of the model used differing in maps and conditions. The double vertical line delimits the models used for no-knot TrmD and structure with PDB code 5wyr only (on the left of the line).}
    \label{tab:methods-maps}
\end{table}

All the models were used in case of the model of TrmD protein.
In case of the no-knot structure of TrmD and the homologue with PDB code 5wyr, only the models *A, *B, and *C (9 models out of 11) were used.

The pulling speed was equal $0.0001 nm/\tau$ and the force spring constant was equal to $0.3 \epsilon/nm^2$

%%%% software %%%%
\section{The software prepared}
\label{sec:methods_software}
The analysis of the data could not be possible without specialized software.
Therefore, as a byproduct of the project, a standalone Python3 package named StretchMe was created.
All the plots and the analysis presented in this report were created using this package.
The sample code for analysis of the traces and whole experiments is provided in App.\ref{ch:code}.

Stretchme was uploaded to the PyPi server, therefore it can be installed by typing \textit{pip3 install stretchme} in the terminal.
It features the full documentation and short manual, available at \textit{https://jsulkowska.cent.uw.edu.pl/pawel/stretchme}.

In particular, StretchMe can analyze single traces and merge them into one experiment.
It can deal with results obtained from experiment as well as from simulations, with, or without linker.
In particular, it can also deal with the case, when the protein gets stretched simultaneously with linker.
Additionaly, StretchMe can also simulate the traces, i.e.\ when given expected persistence lengths, contour lengths and elasticity, Stretchme can provide expected $F(d)$ dependency.
To convenience of the user, Stretchme can read the .xlsx or .csv files.
The data can be given as the path to the file, or as Pandas Dataframe.