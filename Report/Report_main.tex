
\documentclass[11pt]{article}

\usepackage{graphicx}       % to include graphics
\usepackage{hyperref}       % to simplify the use of \href
\usepackage{amsmath}
\usepackage{bm}

\newtheorem{lemma}[theorem]{Lemma}

\title{Report on knotted protein stretching}
\author{Pawel Dabrowski-Tumanski}
\date{\today}

\begin{document}
\maketitle

%%%%%%%% Introduction %%%%%%%%
\section*{Introduction}
\label{sec:intro}
The aim of this work was to analyze the data on stretching of complex topology proteins -- $3_1$ knotted TrmD, Tm1510 and their fusion, and find the best theoretical model allowing recreation of experimental results.
However, the experimental data could not be fitted reasonably with any function commonly used in analysis of protein stretching.
The problem was the simultaneous stretching of both protein and DNA linker.
This led to the necessity of thorough studies of the underlying model and a development of a new technique dealing with a problematic case.
The description of the Worm-Like Chain (WLC model), along with the new technique is contained in Sec.\ref{sec:WLC}.
The resulting analysis of the experimental data is presented in Sec.\ref{sec:exp}.

On the other hand, the need to determine, which model corresponds to the experiment best, various numerical factors had to be calculated \textit{in silico} and compared with their experiment-based counter parts.
However, usually, the theoretical analysis of protein stretching is compared only roughly with the experiment, as the resulting curves are noticeably different.
The reason lies in treating the residue-residue bond as a harmonic spring, which in principle can get stretched infinitely upon force.
Therefore, the correct way of dealing with \textit{in silico} stretching is to include the elastic factor compensating for the artificial stretching of the protein backbone.
The results of such analysis are described in Sec.\ref{sec:theory}.

Finally, such analysis would not be possible if not the software created specially for this purpose.
As the idea of the project was, that it would be passed on at some time, the assertion of the code was, that it has to be understandable and easy-operable for even unexperienced users.
As a result, a standalone package-like code was created, fully described, with the whole change history based on GitHub.
Therefore, the code, upon some standard operations can be finished as a fully operational user-friendly package.

%%%%%%%% Theory of WLC %%%%%%%%
\section*{Worm-like chain model - theory and development}
\label{sec:WLC}
The Worm-Like Chain (WLC) is a model describing a semi-flexible polymer.
Introduced in 1949\cite{kratky1949rontgenuntersuchung} is also called by the surnames of its creators the Kratky\–Porod model.

\subsection*{Freely jointed chain}
\label{subsec:wlc_flc}
One of the simplest model of polymers is the freely-jointed chain (FLC).
In this model, the polymer is composed of $N$ stiff rods, each of the length $l$.
The total length of the polymer is then $Nl=L$.
The orientations of the rods are completely independent of each other, therefore the mean vector connecting the termini $<\vv{\mn{R}} = <\sum_{n=1}^N \vv{\mn{r_i}}> = \sum_{n=1}^N <\vv{\mn{r_i}}> = 0$.
This does not mean, that the ends coincide, rather they fluctuate around each other (``symmetrically'').

\subsection*{WLC model and its assumptions}
\label{subsec:wlc_assumptions}
In the WLC model is a generalization of the FLC model, with the rod length $l\rightarrow 0$ and number of beads $N\rightarrow\infty$, keeping $Nl=L$.
In such case, the polymer may be parametrized by a space curve $\vv{\bm{r}}(s)$ with $s\in(0,L)$.
The conformational space of the WLC polymer is restricted by the energy landscape following from the energy of bending.
In the FJC-like case, the bending energy is given by:
\begin{equation}
    \label{eq:wlc_assumptions_discrete}
    H = -\epsilon\sum_{i=1}^{N-1}\vv{\bm{r}}_{i+1}\cdot\vv{\bm{r}}_{i}
\end{equation}

From the law of cosines (Sec.\ref{subsec:app_cosines}) the product \vv{\bm{r}}_{i+1}\cdot\vv{\bm{r}}_{i} can be rewritten as:

\begin{equation}
    \label{eq:wlc_assumptions_cosines}
    \vv{\bm{r}}_{i+1}\cdot\vv{\bm{r}}_{i} = l^2 - \frac{1}{2}(\vv{\bm{r}}_{i+1}-\vv{\bm{r}}_{i})^2
\end{equation}

With $l$ being the length of the interval.
When passing to the limit, the difference between the position of the beads turn into the tangent vector to the curve:

\begin{equation}
    \label{eq:wlc_assumptions_tangent}
    lim_{l\rightarrow 0}\left( \frac{\vv{\bm{r}}_{i+1}-\vv{\bm{r}}_{i}}{l} \right) = \frac{\partial\}{}}
\end{equation}
the polymer is treated as a smooth, flexible, inextensible rod with a finite maximum length $L_0$.
The polymer may be parametrized by a space curve $\vv{\bm{r}}(s)$ with $s\in(0,L_0)$.


\subsection*{The force-extension dependence in WLC model}
\label{subsec:wlc_curve}
When stretched with force $F$, the conformation space of the chain reduces, which in terms of hamiltonian can be described by the addition of the entropic contribution:

\begin{equation}
    \label{eq:wlc_curve_hamiltonian}
    H = H_{elastic} + H_{entropic} = \frac{1}{2}k_{B}T\int_{0}{L}P\cdot \left( \frac{\partial^2 \vv{\bm{r}}(s)}{\partial ^2 s} \right) ds +d\cdot F
\end{equation}

with $d$ being the extension of the chain.
The analytic solution $F(d)$ to these equation is unknown.
The extension $d$ when no force is applied ($F=0$), and for very small forces, the increase should be linear.
On the other hand, for large forces, with $x\rightarrow L$, one can show, that the extension scales as the inverse of the square root of the force (App.\ref{subsec:app_marko_siggia})
\begin{equation}
    \label{eq:wlc_scalling}
    x = \frac{d}{L}\sim\frac{1}{\sqrt{F}}
\end{equation}
This leads to the interpolation formula, first given by Marko and Siggia\cite{marko1995statistical}, derived in App.\ref{subsec:app_marko_siggia}:

\begin{equation}
    \label{eq:wlc_markosiggia}
    f(x) = \frac{1}{4(1-x)^2} - \frac{1}{4} + x
\end{equation}

This formula is used extensively in the analysis of protein stretching.
One has to note, however, that this formula is approximates the real solution with up to 15\% error for intermediate extensions $x\simeq 0.5$ \cite{petrosyan2017improved}.
Various modifications were introduced to lower the error.
In particular

It can be improved with additional factor:

\begin{equation}
    \label{eq:wlc_improved}
    f(x) = \frac{1}{4(1-x)^2} - \frac{1}{4} + x - 0.8x^{2.15}
\end{equation}

This formula gives the result with around 1\% of the error\cite{petrosyan2017improved}.

\subsection*{Extensible WLC}
\label{subsec:wlc_ewlc}
The WLC model applies to the inextensible chains.
However, some polymers, like DNA, exhibit some extension of the effective bond-length upon the force.
To account on this effect, the extensible WLC model was introduced\cite{wang1997stretching}.

\subsection*{Inverting the force-extension dependence}
\label{subsec:wlc_inverting}

\begin{equation}
    \label{eq:wlc_inverting_polynomial}
    x^3 - (\frac{9}{4}+\frac{F}{p_T})x^2 + (\frac{3}{2}+\frac{2F}{p_T})x - \frac{F}{p_T} = 0
\end{equation}


\begin{equation}
    \label{eq:wlc_inverting_ewlc}
    x^3 - \left(F\cdot(\frac{3}{k} + \frac{1/p}) + \frac{9}{4}\right)x^2 + \left(F^2(\frac{3}{k^2}+\frac{2}{kP}) + F(\frac{9}{2k} + \frac{2}{p}) + \frac{3}{2}\right)x - \left( F^3 (\frac{1}{k^3}+\frac{1}{k^{2}p}) + F^2 (\frac{9}{4k^2} + \frac{2}{kp}) + F(\frac{3}{2k} + \frac{1}{P})\right)\frac{F}{p_T} = 0
\end{equation}

\subsection*{Finding contour length}
\label{subsec:wlc_fitting}


\subsection*{Generalization of Puchner's analysis}
\label{subsec:wlc_method}

%%%%%%%% Results - experiment %%%%%%%%
\section*{Experimental results}
\label{sec:exp}

\subsection*{The proteins analyzed}
\label{subsec:exp_proteins}
TrmD is the $+3_1$-knotted tRNA (guanine-N(1)-)-methyltransferase, from \textit{Haemophilus influenzae} bacteria, classified as a member of SPOUT Methyltransferase superfamilly.
Tm1510 is the protein of unknown function from \textit{Thermotoga maritima} bacteria.
Due to its sequential similarity (?) it is also putatively $+3_1$-knotted protein.

\subsection*{The fitting procedure}

%%%%%%%% Results - theory %%%%%%%%
\section{Analyzing the simulated stretching}
\label{sec:theory}
\subsetion*{The models}


%%%%%%%% Appencices %%%%%%%%
\section*{Appendices}
\label{sec:app}
\subsection*{The law of cosines}
\label{subsec:app_cosines}
\subsection*{The Marko-Siggia derivation}
\label{subsec:app_marko_siggia}
\subsection*{The probability distributions}
\label{subsec:app_probability}
\subsection*{The Dudko-Hummer-Szabo analysis}
\label{subsec:app_dudko}

%%%%%%%% Conclusions %%%%%%%%
\section{Conclusions}
\label{sec:conclusions}

%%%%%%%% References %%%%%%%%
\bibliography{biblio}
\bibliographystyle{ieeetr}
\end{document}


\subsection{Inverting the WLC model}
\label{subsec:inverting}
As the temperature and the persistence length are constant during stretching, and extension $d$ is necessarily smaller than contour length $L$, the WLC model of eq.\ref{eq:wlc} may be simplified to:
\begin{equation}
    \label{eq:wlc_simple}
    F(x) = p_T\left(\frac{1}{4(1-x)^2}-\frac{1}{4}+x\right)
\end{equation}

with $0\lt x \lt 1$ and $p_T=\frac{k_{B}T}{p}$.
Such equation can be used to invert this relation.
In particular, one can rearrange the eq.\ref{eq:wlc_simple} into:




