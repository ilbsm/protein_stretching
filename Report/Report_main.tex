\documentclass[11pt]{article}

\usepackage{graphicx}       % to include graphics
\usepackage{hyperref}       % to simplify the use of \href
\usepackage{amsmath}
\usepackage{bm}
\graphicspath{{./figures/}}

\newtheorem{lemma}[theorem]{Lemma}

\title{Report on knotted protein stretching}
\author{Pawel Dabrowski-Tumanski}
\date{\today}

\begin{document}
\maketitle

%%%%%%%% Introduction %%%%%%%%
\section*{Introduction}
\label{sec:intro}
The aim of this work was to analyze the data on stretching of complex topology proteins -- $3_1$ knotted TrmD, Tm1510 and their fusion (Fig.\ref{fig:intro_proteins}.
Alongside, the analysis of the simulated stretching of those proteins were analyzed in order to select the theoretical model which agrees best with the experiment.
Interestingly, the common approach of fitting the WLC model did not succeed in this case.
In experiment, the problem was caused by the simultaneous stretching of both protein and DNA linker.
On the other hand, in the simulated stretching, the bond joining the protein residues is subjected for stretching, which is not assumed in the standard WLC model.
These problems required development of a new approach to analyze both the experimental, and simulated data.
The extensive description of the Worm-Like Chain (WLC model), along with its developments are contained in Sec.\ref{sec:wlc}.

Overcoming the mentioned problems opened the way for the analysis of the data. The analysis of the experimental data led to a surprising result, that the proteins, previously thought to contain the knot, may actually not be knotted.
This results, along with the discussion of possible cause of this effect is the topic of Sec.\ref{sec:exp}.

On the other hand, the experimental data characterizing the protein unfolding upon stretching allowed for the selection of the theoretical model predicting the protein stretching in the most reliable way. These results are contained in Sec.\ref{sec:theory}.

During realization of this project, the underlying thought was, that it would be taken over possibly by a student not confident enough to start own analysis from scratch.
Therefore, all the tools developed during this project, along with the implementation of the new technique was wrapped into a Python3 package, called Stretchme.
Right now, Stretchme is fully operational package, downloadable from GitHub and PyPi.
The details of the package are contained in Sec.\ref{sec:stretchme} and in the Manual for the package.

Finally, the Sec.\ref{sec:literature} contains literature important in the understanding of stretching of topologically non-trivial proteins.

%%%%%%%% Theory of WLC %%%%%%%%
\section*{Theoretical introduction}
\label{sec:wlc}
The polymeric properties of proteins and DNA are commonly described in terms of variants of Worm-Like Chain (WLC) model.
Here, keeping in mind, that these notes may be useful for people with no prior experience in protein stretching, the complete theory of the WLC model, starting from the very basics is described.

\subsection*{Freely jointed chain}
\label{subsec:wlc_flc}
Before diving into subtleties of WLC model, one may warm up with even simpler freely-jointed chain (FLC) model.
In this model, the polymer is presented as a set of $N$ stiff rods, connected one to another in linear fashion, each of equal length $l$.
The total length of the polymer is then $Nl=L$.
The orientations of the rods are completely independent of each other, therefore the rods are ``freely-jointed''.
This results in the expected end-to-end distance, which is equal:

\begin{equation}
    <\vv{\mn{R}} = <\sum_{n=1}^N \vv{\mn{r_i}}> = \sum_{n=1}^N <\vv{\mn{r_i}}> = 0
\end{equation}
However, this does not mean, that the ends coincide, rather they fluctuate around each other (``symmetrically'').
These fluctuations may be described by the expected length of the vector joining the termini:

The more interesting behaviour of the chain can be observed, if one adds the bending energy on the chain:
\begin{equation}
    \label{eq:wlc_assumptions_discrete}
    H = -\epsilon\sum_{i=1}^{N-1}\vv{\bm{r}}_{i+1}\cdot\vv{\bm{r}}_{i}
\end{equation}

\subsection*{WLC model and its assumptions}
\label{subsec:wlc_assumptions}
The Worm-Like Chain model is a generalization of the FLC model, describing a semi-flexible polymer.
It is called also Kratky\–Porod model because of its creators \cite{kratky1949rontgenuntersuchung}.
One can obtain obtain the WLC model from the FJC with the bending energy term.
Indeed, from the law of cosines (Sec.\ref{subsec:app_cosines}) the product \vv{\bm{r}}_{i+1}\cdot\vv{\bm{r}}_{i} can be rewritten as:

\begin{equation}
    \label{eq:wlc_assumptions_cosines}
    \vv{\bm{r}}_{i+1}\cdot\vv{\bm{r}}_{i} = l^2 - \frac{1}{2}(\vv{\bm{r}}_{i+1}-\vv{\bm{r}}_{i})^2
\end{equation}

With $l$ being the length of the interval.
When passing to the limit, the difference between the position of the beads turn into the tangent vector to the curve:

\begin{equation}
    \label{eq:wlc_assumptions_tangent}
    lim_{l\rightarrow 0}\left( \frac{\vv{\bm{r}}_{i+1}-\vv{\bm{r}}_{i}}{l} \right) = \frac{\partial\}{}}
\end{equation}
the polymer is treated as a smooth, flexible, inextensible rod with a finite maximum length $L_0$.
The polymer may be parametrized by a space curve $\vv{\bm{r}}(s)$ with $s\in(0,L_0)$.


\subsection*{The force-extension dependence in WLC model}
\label{subsec:wlc_curve}
When stretched with force $F$, the conformation space of the chain reduces, which in terms of hamiltonian can be described by the addition of the entropic contribution:

\begin{equation}
    \label{eq:wlc_curve_hamiltonian}
    H = H_{elastic} + H_{entropic} = \frac{1}{2}k_{B}T\int_{0}{L}P\cdot \left( \frac{\partial^2 \vv{\bm{r}}(s)}{\partial ^2 s} \right) ds +d\cdot F
\end{equation}

with $d$ being the extension of the chain.
The analytic solution $F(d)$ to these equation is unknown.
The extension $d$ when no force is applied ($F=0$), and for very small forces, the increase should be linear.
On the other hand, for large forces, with $x\rightarrow L$, one can show, that the extension scales as the inverse of the square root of the force (App.\ref{subsec:app_marko_siggia})
\begin{equation}
    \label{eq:wlc_scalling}
    x = \frac{d}{L}\sim\frac{1}{\sqrt{F}}
\end{equation}
This leads to the interpolation formula, first given by Marko and Siggia\cite{marko1995statistical}, derived in App.\ref{subsec:app_marko_siggia}:

\begin{equation}
    \label{eq:wlc_markosiggia}
    f(x) = \frac{1}{4(1-x)^2} - \frac{1}{4} + x
\end{equation}

This formula is used extensively in the analysis of protein stretching.
One has to note, however, that this formula is approximates the real solution with up to 15\% error for intermediate extensions $x\simeq 0.5$ \cite{petrosyan2017improved}.
Various modifications were introduced to lower the error.
In particular

It can be improved with additional factor:

\begin{equation}
    \label{eq:wlc_improved}
    f(x) = \frac{1}{4(1-x)^2} - \frac{1}{4} + x - 0.8x^{2.15}
\end{equation}

This formula gives the result with around 1\% of the error\cite{petrosyan2017improved}.

\subsection*{Extensible WLC}
\label{subsec:wlc_ewlc}
The WLC model applies to the inextensible chains.
However, some polymers, like DNA, exhibit some extension of the effective bond-length upon the force.
To account on this effect, the extensible WLC model was introduced\cite{wang1997stretching}.

\subsection*{Inverting the force-extension dependence}
\label{subsec:wlc_inverting}

\begin{equation}
    \label{eq:wlc_inverting_polynomial}
    x^3 - (\frac{9}{4}+\frac{F}{p_T})x^2 + (\frac{3}{2}+\frac{2F}{p_T})x - \frac{F}{p_T} = 0
\end{equation}


\begin{equation}
    \label{eq:wlc_inverting_ewlc}
    x^3 - \left(F\cdot(\frac{3}{k} + \frac{1/p}) + \frac{9}{4}\right)x^2 + \left(F^2(\frac{3}{k^2}+\frac{2}{kP}) + F(\frac{9}{2k} + \frac{2}{p}) + \frac{3}{2}\right)x - \left( F^3 (\frac{1}{k^3}+\frac{1}{k^{2}p}) + F^2 (\frac{9}{4k^2} + \frac{2}{kp}) + F(\frac{3}{2k} + \frac{1}{P})\right)\frac{F}{p_T} = 0
\end{equation}

\subsection*{Finding contour length}
\label{subsec:wlc_fitting}


\subsection*{Generalization of Puchner's analysis}
\label{subsec:wlc_method}


\subsection*{The expected chain reduction upon knotting}
\label{subsec:wlc_chain-reduction}

%%%%%%%% Results - experiment %%%%%%%%
\section*{Experimental results}
\label{sec:exp}

\subsection*{The proteins analyzed}
\label{subsec:exp_proteins}
TrmD is the $+3_1$-knotted tRNA (guanine-N(1)-)-methyltransferase, from \textit{Haemophilus influenzae} bacteria, classified as a member of SPOUT Methyltransferase superfamilly.
Tm1510 is the protein of unknown function from \textit{Thermotoga maritima} bacteria.
Due to its sequential similarity (?) it is also putatively $+3_1$-knotted protein.

\subsection*{Presence of the knot}
\label{subsec:exp_knot}

\subsection*{The contour lengths and stretching pathways}
\label{subsec:exp_contour-lengths}

\subsection*{The thermodynamics}
\label{subsec:exp_thermodynamics}

%%%%%%%% Results - theory %%%%%%%%
\section{Analyzing the simulated stretching}
\label{sec:theory}
\subsection*{The models}
\label{subsec:theory_models}

\subsection*{Results for each model}
\label{subsec:theory_results}

\subsection*{Selection of the best model}
\label{subsec:theory_best}

%%%%%%%% Stretchme %%%%%%%%
\section{Stretchme - the Python3 package to analyze protein stretching}
\label{sec:stretchme}
\subsection*{The package functions}
\label{subsec:stretchme_functions}

\subsection*{Installation, availability, documentation}
\label{subsec:stretchme_documentation}

%%%%%%%% literature %%%%%%%%
\section{Literature}
\label{sec:literature}


%%%%%%%% Conclusions %%%%%%%%
\section{Conclusions}
\label{sec:conclusions}


%%%%%%%% Appendices %%%%%%%%
\section*{Appendices}
\label{sec:app}
\subsection*{The law of cosines}
\label{subsec:app_cosines}
In any triangle with the edges and angles as marked in the Fig.\ref{fig:app_cosines}, the following law is fulfilled:
\begin{equation}
c^{2}=a^{2}+b^{2}-2ab\cos \gamma
\end{equation}
\begin{equation}
c=a\cos \beta +b\cos \alpha
c^{2}=ac\cos \beta +bc\cos \alpha
a^{2}=ac\cos \beta +ab\cos \gamma
b^{2}=bc\cos \alpha +ab\cos \gamma
a^{2}+b^{2}=ac\cos \beta +bc\cos \alpha +2ab\cos \gamma
 a^{2}+b^{2}-c^{2}=ac\cos \beta +bc\cos \alpha +2ab\cos \gamma -(ac\cos \beta +bc\cos \alpha
\subsection*{The Marko-Siggia derivation}
\label{subsec:app_marko_siggia}
\subsection*{The probability distributions}
\label{subsec:app_probability}
\subsubsection*{Transformation of probability distributions}


\subsubsection*{Ratio distribution of two normal ditributions}
Let $X, Y\sim\mathcal{N}(\0,\,\1)$ be independent distributions.
Their joint distribution is described with the density:
\begin{equation}
p_{X,Y}(x,y)={\frac {1}{2\pi }}\exp(-{\frac {x^{2}}{2}})\exp(-{\frac {y^{2}}{2}})
\end{equation}
Then their quotient $Z=X/Y$ has the density
\begin{equation}
p_{Z}(z)=\int _{{-\infty }}^{{+\infty }}|y|\,p_{{X,Y}}(zy,y)\,dy=
={\frac {1}{2\pi }}\int _{-\infty }^{\infty }\,|y|\,\exp(-{\frac {(zy)^{2}}{2}})\,\exp(-{\frac {y^{2}}{2}})\,dy
={\frac {1}{2\pi }}\int _{-\infty }^{\infty }\,|y|\,\exp(-{\frac {y^{2}(z^{2}+1)}{2}})\,dy
\end{equation}
Given the integral:
\begin{equation}
\int _{0}^{\infty }\,x\,\exp(-cx^{2})dx={\frac {1}{2c}}
\end{equation}
one easily gets:
\begin{equation}
p_{Z}(z)={\frac {1}{\pi (z^{2}+1)}}
\end{equation}
\subsection*{The Pade approximants}
\label{subsec:app_pade}
\subsection*{The Dudko-Hummer-Szabo analysis}

\label{subsec:app_dudko}
\subsection*{The table of all results}
\label{subsec:app_results}

%%%%%%%% References %%%%%%%%
\bibliography{biblio}
\bibliographystyle{ieeetr}
\end{document}


\subsection{Inverting the WLC model}
\label{subsec:inverting}
As the temperature and the persistence length are constant during stretching, and extension $d$ is necessarily smaller than contour length $L$, the WLC model of eq.\ref{eq:wlc} may be simplified to:
\begin{equation}
    \label{eq:wlc_simple}
    F(x) = p_T\left(\frac{1}{4(1-x)^2}-\frac{1}{4}+x\right)
\end{equation}

with $0\lt x \lt 1$ and $p_T=\frac{k_{B}T}{p}$.
Such equation can be used to invert this relation.
In particular, one can rearrange the eq.\ref{eq:wlc_simple} into:




